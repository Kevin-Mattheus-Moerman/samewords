%this tex file was auto produced from TEI by lombardpress-print on 2017-05-22T09:50:05.379+02:00 using the file:/Users/michael/Documents/coding/SCTA/xslt/lbp-print-xslt/1.0.0/modules/critical/preamble.xsl
\documentclass[a4paper, 12pt]{book}

% imakeidx must be loaded beore eledmac
\usepackage{imakeidx}
\usepackage{libertine}
\usepackage{csquotes}

\usepackage{geometry}
\geometry{left=4cm, right=4cm, top=3cm, bottom=3cm}

\usepackage{fancyhdr}
% fancyheading settings
\pagestyle{fancy}

% latin language
\usepackage{polyglossia}
\setmainlanguage{english}
\setotherlanguage{latin}

% a critical mark
\usepackage{amssymb}

% git package
\usepackage{gitinfo2}


% title settings
\usepackage{titlesec}
\titleformat{\chapter}{\normalfont\large\scshape}{\thechapter}{50pt}{}
\titleformat{\section}{\normalfont\scshape}{\thesection}{1em}{}
\titleformat{\subsection}[block]{\centering\normalfont\itshape}{\thesubsection}{}{}
\titlespacing*{\subsection}{20pt}{3.25ex plus 1ex minus.2 ex}{1.5ex plus.2ex}[20pt]

% reledmac settings
\usepackage[final]{reledmac}

\Xinplaceoflemmaseparator{0pt} % Don't add space after nolemma notes.
\Xlemmadisablefontselection[A] % In fontium lemmata, don't copy font formatting.
\Xarrangement{paragraph}
\linenummargin{outer}
\sidenotemargin{inner}
\lineation{page}

\Xendbeforepagenumber{p.~}
\Xendafterpagenumber{,}
\Xendlineprefixsingle{l.~}
\Xendlineprefixmore{ll.~}

\Xnumberonlyfirstinline[]
\Xnumberonlyfirstintwolines[]
\Xbeforenotes{\baselineskip}

% This should prevent overfull vboxes
\AtBeginDocument{\Xmaxhnotes{0.5\textheight}}
\AtBeginDocument{\maxhnotesX{0.5\textheight}}

\Xprenotes{\baselineskip}

\let\Afootnoterule=\relax
\let\Bfootnoterule=\relax

% other settings
\linespread{1.1}



% custom macros
\newcommand{\name}[1]{#1}
\newcommand{\lemmaQuote}[1]{\textsc{#1}}
\newcommand{\worktitle}[1]{\textit{#1}}
\newcommand{\supplied}[1]{⟨#1⟩}
\newcommand{\suppliedInVacuo}[1]{$\ulcorner$#1$\urcorner$}
\newcommand{\secluded}[1]{{[}#1{]}}
\newcommand{\metatext}[1]{<#1>}
\newcommand{\hand}[1]{\textsuperscript{#1}}
\newcommand{\del}[1]{[#1 del. ms]}
\newcommand{\no}[1]{\emph{#1}\quad}
\newcommand{\corruption}[1]{\textdagger#1\textdagger}

\begin{document}
\fancyhead{}
\fancyfoot[C]{\thepage}
\fancyhead[R]{Quaestiones in libros De anima}
\fancyhead[L]{Johannes Dinsdale}

\fancyhead[C]{DRAFT}


\chapter*{Johannes Dinsdale: Quaestiones in libros De anima}
\addcontentsline{toc}{chapter}{Quaestiones in libros De anima}


\begin{latin}
\beginnumbering
\pstart[\subsection*{\metatext{Quaestio 1: Utrum de anima possit nobis acquiri scientia}}]
\edlabelS{da-49-l1q1-274hkz}%
\ledsidenote{B 148va}\ledsidenote{O 164rb}%
\edtext{Item}{\lemma{Item}\Bfootnote{\emph{om.} B}} quaeratur\edtext{}{\lemma{}\Bfootnote[nosep]{nunc \emph{post} quaeratur B}} primo utrum de anima possit nobis acquiri scientia.
\edlabelE{da-49-l1q1-274hkz}
\pend

\medbreak

\pstart[\medbreak{}]
\edlabelS{da-49-l1q1-j01jdw}%
\no{1}
Videtur quod non.
\edlabelE{da-49-l1q1-j01jdw}
\pend

\pstart
\edlabelS{da-49-l1q1-ysmgk1}%
\no{1.1}
Illud de quo \sameword{est} scientia \sameword{est} intelligibile, quia cum scientia sit habitus intellectus, de quo \sameword{est} scientia oportet esse intelligibile; sed anima non \sameword{est} intelligibile, quia omnis nostra cognitio ortum habet a sensu, \edtext{unde ipsum intelligere non \sameword[1]{est}}{\lemma{unde \dots{} \sameword{est}}\applabel{da-49-l1q1-app-pa6tn6}\Bfootnote{quia nihil intelligimus B}} sine phantasmate, sed anima sub sensu non cadit, nec phantasma facit; ergo et cetera.
\edlabelE{da-49-l1q1-ysmgk1}
\pend

\pstart
\edlabelS{da-49-l1q1-mjzkyp}%
\no{1.2}
Praeterea, unum et idem non potest esse simul movens et motum, quia\edtext{}{\lemma{}\Bfootnote[nosep]{si \emph{post} quia B}} sic \edtext{idem esset}{\lemma{idem esset}\Bfootnote{\emph{inv.} B}} actu et potentia respectu eiusdem; sed \sameword{cognitum} est movens respectu cognocentis; ergo unum et idem non potest esse \edtext{cognoscens}{\lemma{cognoscens}\applabel{da-49-l1q1-app-18nkzl}\Bfootnote{movens B}} et \edtext{\sameword[1]{cognitum}}{\lemma{\sameword{cognitum}}\applabel{da-49-l1q1-app-87ko58}\Bfootnote{motum B}},\edtext{}{\lemma{}\Bfootnote[nosep]{sed \emph{post} motum B}} hoc tamen contingeret si de anima \edtext{esset scientia}{\lemma{esset scientia}\applabel{da-49-l1q1-app-hptxkl}\Bfootnote{cognitionem haberemus B}}.
\edlabelE{da-49-l1q1-mjzkyp}
\pend

\pstart
\edlabelS{da-49-l1q1-mjgjs9}%
\no{1.3}
Praeterea, \edtext{sicut oculus \edtext{\sameword[2]{nycticoracis}}{\lemma{\sameword{nycticoracis}}\applabel{da-49-l1q1-app-6ef4va}\Bfootnote{vespertilionis O}} \edtext{\suppliedInVacuo{se habet}}{\lemma{se habet}\applabel{da-49-l1q1-app-qzz9yj}\Bfootnote{\emph{spat. vac. 8 litt.} B; \emph{om.} O}} ad lumen solis, sic intellectus noster ad ea quae sunt manifestissima in natura}{\lemma{}\Afootnote[nosep]{Cf. Arist. \worktitle{Metaph.}\index[works]{Metaphysics} II.1 993b9--11 (νυκτερίς).}}, de quorum numero est anima, \edtext{saltim}{\lemma{saltim}\Bfootnote{\emph{om.} B}} humana; sed oculus \sameword{nycticoracis} non potest apprehendere lumen solis; ergo et cetera.
\edlabelE{da-49-l1q1-mjgjs9}
\pend

\pstart
\edlabelS{da-49-l1q1-xdpfxs}%
\no{1.4}
Praeterea, nostrum intelligere est cum continuo et tempore; sed anima, cum sit indivisibilis et perpetua, nec est continua nec temporalis; ergo et cetera.
\edlabelE{da-49-l1q1-xdpfxs}
\pend

\medbreak

\pstart[\medbreak{}]
\edlabelS{da-49-l1q1-o9whrl}%
\no{2}
Oppositum patet per determinationem Philosophi\index[persons]{Aristotle}.
\edlabelE{da-49-l1q1-o9whrl}
\pend

\medbreak

\pstart[\medbreak{}]
\edlabelS{da-49-l1q1-ywem12}%
\no{3.1}
Dicendum quod cum scientia sit habitus acquisitus per demonstrationem, et ad demonstrationem tria requirantur ( \edtext{scilicet}{\lemma{scilicet}\Bfootnote{\emph{om.} B}} subiectum, passio, et principium per quod ostenditur passio de subiecto), ubi est invenire ista tria, ibi \edtext{contingit ponere scientiam}{\lemma{contingit ponere scientiam}\applabel{da-49-l1q1-app-eje7pm}\Bfootnote{est scientiam ponere B}}. Nunc autem anima quoddam subiectum est cuius sunt \edtext{multae}{\lemma{multae}\Bfootnote{\emph{om.} B}} proprietates et passiones, ut patebit inferius. Sunt etiam principia per quae istae passiones probari possunt de anima. Si enim accipiatur quod quid est animae pro medio, \edtext{per ipsum}{\lemma{per ipsum}\applabel{da-49-l1q1-app-k2lfc9}\Bfootnote{\emph{om.} B}} concludi potest propria passio eius de anima, et ita de anima potest \edtext{aliquid sciri sive}{\lemma{aliquid sciri sive}\Bfootnote{\emph{om.} B}} esse aliqualis scientia.
\edlabelE{da-49-l1q1-ywem12}
\pend

\pstart
\edlabelS{da-49-l1q1-t57fj3}%
\no{3.2}
Praeterea, accidentia non sunt per se entia, sed in alio. Qui ergo cognoscit accidentia, manuduci potest in cognitionem eius cuius sunt. Nunc autem multa accidentia ipsius animae nobis sunt manifesta: Operationes \edtext{enim}{\lemma{enim}\Bfootnote{\emph{om.} B}} artificiales nobis notae sunt, quae \edtext{tamen}{\lemma{tamen}\Bfootnote{\emph{om.} B}} non fiunt absque intelligere, \sameword{et} intelligere procedit ab aliqua potentia, \sameword{et} potentia \edtext{fluit}{\lemma{fluit}\applabel{da-49-l1q1-app-mqmbne}\Bfootnote{\emph{om.} B}} ab essentia; \edtext{\sameword[1]{et} sic est de aliis}{\lemma{\sameword{et} \dots{} aliis}\Bfootnote{eodem modo est de B}} operationibus quae procedunt ab irascibili. Unde per multa \edtext{quae nobis nota sunt}{\lemma{quae nobis nota sunt}\applabel{da-49-l1q1-app-b1mson}\Bfootnote{nobis B}} devenire \edtext{possumus}{\lemma{possumus}\Bfootnote{possunt B}} in cognitionem animae. Quia tamen scire est \edtext{\sameword[1]{per causam}}{\lemma{\sameword{per causam}}\Bfootnote{causam rei B}} cognoscere, \sameword{et} talis \edtext{cognitio de anima}{\lemma{cognitio de anima}\applabel{da-49-l1q1-app-gfllzy}\Bfootnote{\emph{om.} B}} procedit per effectus \edtext{\sameword[1]{et}}{\lemma{\sameword{et}}\Bfootnote{\emph{om.} B}} non \sameword{per causam}, ideo Philosophus\index[persons]{Aristotle} talem cognitionem tradens de anima \edtext{istam cognitionem nominat}{\lemma{istam cognitionem nominat}\Bfootnote{ipsam vocat B}} \edtext{\enquote{historiam}}{\lemma{}\Afootnote[nosep]{Arist. \worktitle{DA}\index[works]{De anima} I.1 402a4.}}. Extensive tamen dici potest scientia.
\edlabelE{da-49-l1q1-t57fj3}
\pend

\medbreak

\pstart[\medbreak{}]
\edlabelS{da-49-l1q1-1mle08}%
\no{Ad 1.1}
Ad primum argumentum dicendum \edtext{quod minor est falsa}{\lemma{quod minor est falsa}\Bfootnote{per interemptionem minoris B}}. \edtext{Et}{\lemma{Et}\Bfootnote{\emph{om.} B}} ad probationem dicendum quod \edtext{aliquid}{\lemma{aliquid}\applabel{da-49-l1q1-app-yl5yi8}\Bfootnote{aliquod B}} cadit |\ledsidenote{B 148vb} in sensu dupliciter: aut \sameword{per} positionem aut \edtext{\sameword[1]{per}}{\lemma{\sameword{per}}\Bfootnote{\emph{om.} O}} privationem. Per privationem \sameword{sicut} tenebra et indivisibilia, ut punctum et unitas. Per positionem contingit dupliciter, aut \sameword{per} speciem sui, aut \sameword{per} speciem alterius; \sameword{per} speciem sui \edtext{\sameword[1]{sicut}}{\lemma{\sameword{sicut}}\Bfootnote{ut B}} color \sameword{videtur}, \edtext{per speciem alterius \edtext{\sameword[2]{sicut}}{\lemma{\sameword{sicut}}\Bfootnote{ut B}} \edtext{\sameword[2]{videtur}}{\lemma{\sameword{videtur}}\applabel{da-49-l1q1-app-tjx5f0}\Bfootnote{\emph{om.} B}} Diari\index[persons]{Diares} filius}{\lemma{}\Afootnote[nosep]{Arist. \worktitle{DA}\index[works]{De anima} II.6 418a20--22.}}. Unde, licet anima non cadat sub sensu per \edtext{\sameword[1]{se}}{\lemma{\sameword{se}}\applabel{da-49-l1q1-app-o6vevk}\Bfootnote{rei B}}, \edtext{cadit}{\lemma{cadit}\applabel{da-49-l1q1-app-r2io0f}\Bfootnote{cadat B O}} tamen sub sensu per alterum, ut per sui effectus, \edtext{et}{\lemma{et}\Bfootnote{\emph{om.} B}} eodem modo, licet per \sameword{se} phantasma non faciat, \edtext{aliud}{\lemma{aliud}\Bfootnote{aliquid O B}} tamen phantasma facit, quod in eius cognitionem ducere potest.
\edlabelE{da-49-l1q1-1mle08}
\pend

\pstart
\edlabelS{da-49-l1q1-i9rohp}%
\no{Ad 1.2}
Ad \edtext{aliud}{\lemma{aliud}\Bfootnote{secundum B}} dicendum quod dupliciter dicitur motus: uno modo est actus imperfecti, \edtext{\sameword[1]{sicut}}{\lemma{\sameword{sicut}}\Bfootnote{et sic O}} definitur \edtext{a Philosopho\index[persons]{Aristotle} in}{\lemma{a Philosopho in}\Bfootnote{\emph{om.} B}} \edtext{tertio \worktitle{Physicorum}\index[works]{Physics};}{\lemma{}\Afootnote[nosep]{Arist. \worktitle{Phys.}\index[works]{Physics} III.2 201b31--33.}} alio modo est actus perfecti, \edtext{\sameword[1]{sicut}}{\lemma{\sameword{sicut}}\Bfootnote{sic O}} intelligere et \edtext{cognoscere}{\lemma{cognoscere}\applabel{da-49-l1q1-app-o28m47}\Bfootnote{sentire O}} dicuntur motus: Primo modo non potest idem esse movens et motum per se, per accidens tamen nihil prohibet, \sameword{sicut} nauta \edtext{movet navem per se}{\lemma{movet navem per se}\Bfootnote{per se \emph{ante} movet navem \emph{scr.} B}}, \edtext{qua mota movet seipsum}{\lemma{qua mota movet seipsum}\Bfootnote{et motu navi movetur B}}\edtext{}{\lemma{}\applabel{da-49-l1q1-app-3bz2f8}\Bfootnote[nosep]{per accidens \emph{post} movetur B}}. Secundo modo nihil prohibet idem \edtext{movere se ipsum}{\lemma{movere se ipsum}\applabel{da-49-l1q1-app-633yrw}\Bfootnote{esse movens et motum respectu sui ipsius B}}. Sed tamen differentia est: aliqua \edtext{enim}{\lemma{enim}\Bfootnote{\emph{om.} B}} est \sameword{substantia} semper actu intelligens, et talis \edtext{\sameword[1]{substantia}}{\lemma{\sameword{substantia}}\Bfootnote{\emph{om.} O}} potest intelligere se per se, \edtext{\sameword[1]{sicut} est de prima causa et intelligentiis}{\lemma{\sameword{sicut} \dots{} intelligentiis}\applabel{da-49-l1q1-app-05821q}\Bfootnote{sed est de prima causa et intelligentiis O; \emph{om.} B}}; sed aliqua est\edtext{}{\lemma{}\applabel{da-49-l1q1-app-bnz3xq}\Bfootnote[nosep]{substantia \emph{post} est B}} non semper actu intelligens, \edtext{\sameword{sicut} est anima humana}{\lemma{sicut est anima humana}\applabel{da-49-l1q1-app-as4rnu}\Bfootnote{\emph{om.} B}}, et talis\edtext{}{\lemma{}\applabel{da-49-l1q1-app-1w0l0f}\Bfootnote[nosep]{substantia \emph{post} talis B}} \edtext{non intelligit}{\lemma{non intelligit}\Bfootnote{non potest intelligere B}} se per se, quia nihil intelligitur nisi secundum quod actu est, et talis substantia, cum sit in potentia intelligens \edtext{non est in actu nisi per alterum, ut per speciem intelligibilem, ideo}{\lemma{non \dots{} ideo}\applabel{da-49-l1q1-app-bc6142}\Bfootnote{non est in actu nisi per alterum, ut per speciem intelligibilem, ideo O; per speciem alterius B}} per alterum potest \sameword{se} intelligere. Per hoc enim quod anima intelligit obiectum per speciem potest intelligere suum actum, et per actum potest reflectere \edtext{\sameword[1]{se}}{\lemma{\sameword{se}}\Bfootnote{\emph{om.} B}} supra suam essentiam; \edtext{unde anima nostra quodammodo intelligit se sicut nauta movet navem}{\lemma{unde \dots{} navem}\applabel{da-49-l1q1-app-99tkjl}\Bfootnote{unde anima nostra quodammodo intelligit se sicut nauta movet navem O; \emph{om.} B}}.
\edlabelE{da-49-l1q1-i9rohp}
\pend

\pstart
\edlabelS{da-49-l1q1-52psh5}%
\no{Ad 1.3}
Ad \edtext{aliud}{\lemma{aliud}\Bfootnote{tertium B}} dicendum quod licet oculus nycticoracis non possit \sameword{apprehendere} directe lumen solis, \edtext{potest tamen indirecte aliquam claritatem \sameword[1]{apprehendere}}{\lemma{potest \dots{} \sameword{apprehendere}}\Bfootnote{potest tamen indirecte aliquam claritatem apprehendere O; aliquem tamen effectum eius potest apprehendere B}}, et si visus \edtext{eius}{\lemma{eius}\Bfootnote{\emph{om.} B}} esset discursivus, \edtext{posset}{\lemma{posset}\applabel{da-49-l1q1-app-ss4j7m}\Bfootnote{possit B}} \edtext{cognoscere}{\lemma{cognoscere}\Bfootnote{intelligere B}} lumen solis. Nunc \edtext{autem, etsi}{\lemma{autem, etsi}\Bfootnote{\emph{om.} B}} intellectus noster\edtext{}{\lemma{}\Bfootnote[nosep]{etsi \emph{post} noster \emph{del.} O}} non \edtext{potest}{\lemma{potest}\applabel{da-49-l1q1-app-d8u1cl}\Bfootnote{posset O; possit B}}\edtext{}{\lemma{}\applabel{da-49-l1q1-app-wajnh7}\Bfootnote[nosep]{directe \emph{post} potest B}} in cognitionem \edtext{perfectam}{\lemma{perfectam}\Bfootnote{\emph{om.} B}} substantiarum separatarum, tamen \edtext{aliqui effectus earum apparent nobis, \edtext{\sameword[2]{per} quos manuducimur in earum notitiam}{\lemma{\sameword{per} \dots{} notitiam}\Bfootnote{\emph{in marg.} O}}}{\lemma{aliqui \dots{} notitiam}\applabel{da-49-l1q1-app-zq4kdv}\Bfootnote{aliqui effectus earum apparent nobis per quos manuducimur in earum notitiam (per
quos manuducimur in earum notitiam \emph{in marg.}) O; potest in effectus earum B}}, et quia intellectus noster est discursivus, ideo \edtext{potest}{\lemma{potest}\Bfootnote{per effectus possumus B}} in \edtext{aliqualem}{\lemma{aliqualem}\applabel{da-49-l1q1-app-yq0rb5}\Bfootnote{aliqualiter O}} cognitionem \edtext{earum ut \sameword{per} effectus}{\lemma{earum ut per effectus}\Bfootnote{substantiarum separatarum B}}. Magis |\ledsidenote{O 164va} tamen cognoscimus de anima quam de \edtext{substantiis}{\lemma{substantiis}\applabel{da-49-l1q1-app-26i4eh}\Bfootnote{aliis O}} separatis, quia effectus \edtext{ipsius}{\lemma{ipsius}\Bfootnote{\emph{om.} B}} animae \edtext{\sameword[1]{nobis apparentes}}{\lemma{\sameword{nobis apparentes}}\applabel{da-49-l1q1-app-li4ovm}\Bfootnote{\emph{om.} B}} magis adaequant virtutem eius quam effectus\edtext{}{\lemma{}\Bfootnote[nosep]{quae \emph{post} effectus O}} substantiarum separatarum \edtext{\sameword[1]{nobis apparentes}}{\lemma{\sameword{nobis apparentes}}\Bfootnote{\emph{om.} B}} \edtext{adaequant virtutem earum}{\lemma{adaequant virtutem earum}\applabel{da-49-l1q1-app-bene1c}\Bfootnote{\emph{om.} O}}.
\edlabelE{da-49-l1q1-52psh5}
\pend

\pstart
\edlabelS{da-49-l1q1-h6ks8n}%
\no{Ad 1.4}
Ad aliud dicendum quod \edtext{intelligere nostrum}{\lemma{intelligere nostrum}\Bfootnote{\emph{inv.} B}} non est sine\edtext{}{\lemma{}\applabel{da-49-l1q1-app-mvv35w}\Bfootnote[nosep]{phantasmate B}} continuo et tempore, quia non est sine phantasmate. Non tamen oportet\edtext{}{\lemma{}\applabel{da-49-l1q1-app-l0y465}\Bfootnote[nosep]{quod \emph{post} oportet B}} omne \edtext{intelligere esse}{\lemma{intelligere esse}\applabel{da-49-l1q1-app-mmea7s}\Bfootnote{quod a nobis est quocumque modo intelligibile sit B}} continuum et \edtext{temporale}{\lemma{temporale}\applabel{da-49-l1q1-app-a7va9c}\Bfootnote{temporalis B}}.
\edlabelE{da-49-l1q1-h6ks8n}
\pend


\endnumbering
\end{latin}
\end{document}
